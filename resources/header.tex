% packages
\usepackage{graphicx}
\usepackage[german]{babel}
\usepackage{setspace}
\usepackage[fleqn]{amsmath}
\usepackage{amsfonts,amssymb}
\usepackage{mathtools}
\usepackage{tikz}
\usepackage[utf8]{inputenc}
\usepackage[left=2cm,right=2cm,top=2cm,bottom=2cm,includeheadfoot]{geometry}
\usepackage{multicol}
\usepackage[misc]{ifsym}
\usepackage{hyperref}
\usepackage{xcolor}
\usepackage{fancyhdr}
\usepackage{tikz}
\usepackage{tasks}
\usepackage{framed}

% style
\fancyhf{} 
\fancyhead[L]{\begin{tabular}{ll}
\href{https://ekvv.uni-bielefeld.de/kvv_publ/publ/vd?id=388066971}{Praktische Übung zur Zeitreihenanalyse} \\
Sommersemester 2023
\end{tabular}
}
\fancyhead[C]{}
\fancyhead[R]{
    \begin{tabular}{r}
    \href{mailto:lennart.oelschlaeger@uni-bielefeld.de}{Lennart Oelschl\"ager \Letter} \\
       14-tägig, Mo, 16:15 -- 17:45, T2-213
    \end{tabular}
}
\renewcommand{\headrulewidth}{0pt} 
\fancyfoot[C]{\thepage}
\renewcommand{\footrulewidth}{0pt} 
\thispagestyle{fancy}
\setlength{\parindent}{0cm}

% commands
\renewcommand{\labelenumi}{\alph{enumi})}
\newcommand{\aufgabenblatt}[1]{
    \begin{center}
    \Large\sffamily
    \textbf{Aufgabenblatt #1}
    \ifcomment
    -- Lösungen
    \fi
\end{center}
}
\newcommand{\aufgabe}[2]{
    \subsection*{Aufgabe #1 (#2)}
}
\newif\ifcomment
\newcommand{\comment}[1]{
    \ifcomment
    \begin{framed}
    \begin{minipage}[l]{0.9\textwidth}
    #1
    \end{minipage}
    \end{framed}
    \fi
}
\newcommand{\inputBlatt}[1]{
    \newpage 
    \setcounter{page}{1} 
    \setcounter{equation}{0} 
    \setcounter{footnote}{0} 
    \input{#1}
}

\newcommand{\betrag}[1]{\left| #1 \right|}

\newcommand{\norm}[1]{\left\lVert #1 \right\rVert}
