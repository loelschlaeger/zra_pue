\aufgabenblatt{3}

\aufgabe{1}{Multiple lineare Regression}

\begin{enumerate}

\item Wie wird das multiple lineare Regressionsmodell definiert, und wie können die Modellparameter geschätzt werden?

\comment{
Eine Variable $y$ wird mit $k$ Regressoren $x_k$ durch $y = \sum_{j = 1}^k x_j \beta_j + u$ erklärt. Der $u$ Term beinhaltet den Modellfehler,  die $\beta_j$'s sind Modellparameter. Es sollen konkrete Werte für die Parameter gefunden werden, die am besten zu gegebenen Daten passen. Die populärste Methode ist der Kleinste-Quadrate-Schätzer: Er bestimmt solche Schätzwerte, sodass die Summe der quadrierten Residuen minimal ist.
}

\item Welche Eigenschaften hat der Kleinste-Quadrate Schätzer, und welche Voraussetzungen müssen dafür erfüllt sein?

\comment{
Unter den Annahmen MLR.1 (lineares Modell), MLR.2 (Zufallsstichprobe), MLR.3 (Information in den Regressoren, keine Multikollinearität) und MLR.4 (bedingte Fehlererwartung ist Null) ist der KQ-Schätzer erwartungstreu. Gilt zusätzlich MLR.5 (Homoskedastie), so ist die Varianz des KQ-Schätzers die minimale Varianz unter allen linearen und erwartungstreuen Schätzern (Gauss-Markov Theorem). Gilt auch MLR.6 (Fehler sind normalverteilt), so ist der KQ-Schätzer normalverteilt. Unter einer technischen Annahme an die Daten ist er konsistent.
}

\item Bitte prognostizieren Sie den $y$ Wert für $x_1 = 6$ und $x_2 = -1$, gegeben die Daten
\begin{align*}
    x_1 &= \begin{pmatrix} 1 & 2 & 3 & 4 & 5 & 1 & 2 & 3 & 4 & 5 \end{pmatrix}, \\
    x_2 &= \begin{pmatrix} 1 & 1 & 2 & 2 & 3 & 3 & 4 & 4 & 5 & 5 \end{pmatrix}, \\
    y   &= \begin{pmatrix} 1 & 3 & 1 & 5 & 2 & -3 & -3 & -1 & -2 & -2 \end{pmatrix}.
\end{align*} 

\comment{
Erstelle die drei Datenvektoren in \texttt{R} und erzeuge daraus einen \texttt{data.frame} mittels \texttt{data <- data.frame(y, x\_1, x\_2)}. Schätze das lineare Modell mit dem Aufruf \texttt{model <- lm(y $\sim$ x\_1 + x\_2, data)} und prognostiziere mit \texttt{model\$coefficients \%*\% c(1, 6, -1)}. Ich erhalte den Wert $12$.
}

\end{enumerate}

\aufgabe{2}{Trendmodellierung durch polynomiale Regression}

\begin{enumerate}

\item Sie finden im \href{https://moodle.uni-bielefeld.de/course/view.php?id=1035}{Lernraum der PÜ} den Datensatz \texttt{hermannslauf\_frauen.csv} mit den Hermannslaufbestzeiten der Frauen. Bitte erstellen Sie eine Grafik der Bestzeiten in Minuten. Können Sie Elemente des Komponentenmodells erkennen und interpretieren?

\comment{
Siehe \texttt{R} Code für die Erstellung der Grafik. Im klassischen additiven Komponentenmodell wird angenommen, dass sich die Beobachtung $x_t$ zum Zeitpunkt $t$ additiv als $x_t = T_t + S_t + Z_t + R_t$ ergibt:
\begin{itemize}
    \item $T_t$ ist der Trendwert bei $t$ (langfristige Veränderung des Erwartungswertes),
    \item $S_t$ ist der saisonale Wert bei $t$ (periodische Schwankungen),
    \item $Z_t$ ist die zyklische Komponente bei $t$ (nicht-periodische Schwankungen),
    \item $R_t$ ist die verbleibende Variation bei $t$ (Restkomponente).
\end{itemize}
Wir erkennen Trend (Zeiten wurden besser, aber scheinen inzwischen gesättigt) und Restkomponente (unbeobachtete Einflüsse, zum Beispiel das Wetter), keine Saisonalitäten oder Zyklen (dafür gäbe es hier auch keine Interpretation).
}

\item Bitte passen Sie polynomiale Trendmodelle verschiedenen Grades mittels linearer Regression an die Daten an. Wie würden Sie den Polynomgrad wählen und warum?

\comment{
Wir betrachten polynomiale Trendmodelle der Form $T_t = \beta_0 + \beta_1 t + \dots + \beta_k t^k$ für verschiedene $k \in \mathbb{N}$. Zur Bestimmung von $k$ können wir F-Tests, Modellselektionskriterien und Kreuzvalidierung betrachten. Siehe \texttt{R} Code für die Anpassungen und Modellselektionen.
}

\item Im Jahr 2005 wurde die Laufstrecke um 500 Meter verlängert. Modellieren Sie an dieser Stelle einen Strukturbruch in der Zeitreihe und testen Sie auf statistische Signifikanz. 

\comment{
Siehe \texttt{R} Code für die Modellierung des Strukturbruchs und Test auf Signifikanz.
}

\end{enumerate}

\aufgabe{3}{Trendbereinigung mittels variate-difference Methode}

Betrachten Sie die folgenden drei Zeitreihen für $t = 1,\dots,T$ mit Restkomponente $u_t$:
\begin{itemize}
    \item $a_t = 3 + u_t$
    \item $b_t = a_t + 0.4t$
    \item $c_t = b_t + 0.3t^2$
\end{itemize}

\begin{enumerate}
    
\item Bitte berechnen Sie jeweils die erste und zweite Differenz der drei Zeitreihen.

\comment{
\begin{itemize}
    \item $\Delta a_t = \Delta u_t$, $\Delta^2 a_t = \Delta^2 u_t$
    \item $\Delta b_t = 0.4 + \Delta u_t$, $\Delta^2 b_t = \Delta^2 u_t$
    \item $\Delta c_t = 0.1 + 0.6t + \Delta u_t$, $\Delta^2 c_t = 0.6 + \Delta^2 u_t$
\end{itemize}
}

\item Die Restkomponente $u_t$ sei eine standardnormalverteilte und unabhängige Zufallsvariable. Welchen Erwartungswert und Varianz haben die Zeitreihen sowie ihre erste und zweite Differenz für gegebenes $t$ jeweils?

\comment{
\begin{itemize}
    \item $\E(a_t) = 3$, $\E(\Delta a_t) = 0$, $\E(\Delta^2 a_t) = 0$
    \item $\Var(a_t) = 1$, $\Var(\Delta a_t) = 2$, $\Var(\Delta^2 a_t) = 6$

    \item $\E(b_t) = 3 + 0.4t$, $\E(\Delta b_t) = 0.4$, $\E(\Delta^2 b_t) = 0$
    \item $\Var(b_t) = 1$, $\Var(\Delta b_t) = 2$, $\Var(\Delta^2 b_t) = 6$

    \item $\E(c_t) = 3 + 0.4t + 0.3t^2$, $\E(\Delta c_t) = 0.1 + 0.6t$, $\E(\Delta^2 c_t) = 0.6$
    \item $\Var(c_t) = 1$, $\Var(\Delta c_t) = 2$, $\Var(\Delta^2 c_t) = 6$
\end{itemize}
}

\item Simulieren Sie die Zeitreihe $c_t$ bis $T = 100$ und berechnen Sie die erste und zweite Differenz. Visualisieren Sie anschließend das Ergebnis.

\comment{
\texttt{T <- 100 \\
t <- 1:T \\
u <- rnorm(T) \\
c <- ts(3 + 0.4*t + 0.3*t\^{}2 + u) \\
plot(c) \\
plot(diff(c)) \\
plot(diff(c, difference = 2))}
}

\end{enumerate}


