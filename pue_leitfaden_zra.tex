\begin{center}
    \Large\sffamily
    \textbf{Leitfaden zur Zeitreihenanalyse}
\end{center}

Im Rahmen der Studienleistung analysieren Sie eine selbstgewählte Zeitreihe, siehe dazu auch das erste Aufgabenblatt im \href{https://moodle.uni-bielefeld.de/course/view.php?id=1035}{Lernraum der PÜ}. Dieses Dokument listet Fragen auf, die Sie mit Ihrer Analyse beantworten können. Die Fragen sind gruppiert nach der Vorlesung, in der Sie die entsprechenden Verfahren kennengelernt haben. Sie werden fortlaufend ergänzt.

\paragraph{Vorlesung 1 (14.04.2023)}
\begin{itemize}
    \item Wer hat Ihren Datensatz erhoben und was für ein Ziel soll wohl damit verfolgt werden?
    \item Wie wählen Sie die Argumente \texttt{start}, \texttt{end} und \texttt{frequency}, wenn Sie in \texttt{R} ein Zeitreihenobjekt mit der Funktion \texttt{ts()} erstellen? Wie können Sie Ihre Zeitreihe grafisch darstellen?
\end{itemize}

\paragraph{Vorlesung 2 (21.04.2023)}
\begin{itemize}
    \item Mit welchen Statistiken können Sie Ihre Zeitreihe beschreiben?
\end{itemize}

\paragraph{Vorlesung 3 (28.04.2023)}
\begin{itemize}
    \item Aus welchen Komponenten besteht das klassische Komponentenmodell und welche davon finden Sie rein visuell in Ihrer Zeitreihe wieder?
    \item Wie können Sie den Trend einer Zeitreihe durch Regression modellieren? 
    \item Wie sieht Ihre Zeitreihe nach einer Trendbereinigung durch Differenzenbildung aus?
\end{itemize}

\paragraph{Vorlesung 4 (05.05.2023)}
\begin{itemize}
    \item Wie wählen Sie den Grad einer polynomialen Trendmodellierung mittels F-Test, Modellselektionskriterien, oder Kreuzvalidierung? Variiert die jeweilige Empfehlung?
    \item Gibt es Strukturbrüche im Trend Ihrer Zeitreihe, die Sie visuell erkennen oder die durch Ereignisse bedingt sind? Wie können Sie diese Strukturbrüche modellieren und auf Signifikanz testen?
\end{itemize}

\paragraph{Vorlesung 5 (12.05.2023)}
\begin{itemize}
    \item Wie können Sie den Trend Ihrer Zeitreihe durch Glättungsverfahren herausstellen, zum Beispiel mit heuristischer oder polynomialer Glättung? 
    \item Wie wählen Sie die Parameter der Glättungsmethoden?
\end{itemize}

\paragraph{Vorlesung 6 (19.05.2023)}
\begin{itemize}
    \item Sie haben auch die exponentielle Glättung kennengelernt, welches Ergebnis bietet dieses Verfahren? Und welche Glättungsparameter schlägt die \texttt{HoltWinters()} Funktion vor?
    \item Welches Ergebnis liefert eine Saisonmodellierung ihrer Zeitreihe mittels Regression auf Basis von Dummyvariablen?
\end{itemize}

\paragraph{Vorlesung 7 (26.05.2023)}
\begin{itemize}
    \item Welches Ergebnis liefert eine Saisonelimination mit Differenzenbildung oder gleitender Glättung (und wie wählen Sie dafür das Fenster und die Gewichte)?
    \item Welches Dekompositionsergebnis für Ihre Zeitreihe liefert die \texttt{decompose()} Funktion?
\end{itemize}

\paragraph{Vorlesung 8 (02.06.2023)}
\begin{itemize}
    \item Wie sieht das empirische Korrelogramm (bzw.\ die \textit{partiellen} Autokorrelationen) aus, nachdem Sie Trend und Saison eliminiert haben? 
    \item Was für eine Vermutung haben Sie damit über die Existenz von Zyklen in Ihrer Zeitreihe?
\end{itemize}

\paragraph{Vorlesung 9 (09.06.2023)}
\begin{itemize}
    \item Weist Ihre empirische Autokorrelation auf einen $\text{MA}(q)$ Prozess hin?
    \item Welches Ergebnis liefert eine Zyklus-Modellierung mittels $\text{MA}(q)$ Prozess?
\end{itemize}

\paragraph{Vorlesung 10 (16.06.2023)}
\begin{itemize}
    \item Weist Ihre empirische partielle Autokorrelation auf einen $\text{AR}(p)$ Prozess hin?
    \item Welches Ergebnis liefert eine Zyklus-Modellierung mittels $\text{AR}(p)$ Prozess? 
\end{itemize}

\paragraph{Vorlesung 11 (23.06.2023)}
\begin{itemize}
    \item Welches Zyklus-Modell empfiehlt das AIC?
    \item Zu welchem Ergebnis kommt ein Signifikanztest auf Autokorrelation in der Restkomponente nach der Zykluselimination?
\end{itemize}

\paragraph{Vorlesung 12 (30.06.2023)}
\begin{itemize}
    \item Wie können Sie den weiteren Verlauf Ihrer Zykluskomponente vorhersagen?
    \item Wie entwickelt sich Ihre Zeitreihe gemäß dem gesamten Komponentenmodell weiter?
\end{itemize}
