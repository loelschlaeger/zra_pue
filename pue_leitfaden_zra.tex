\begin{center}
    \Large\sffamily
    \textbf{Leitfaden zur Zeitreihenanalyse}
\end{center}

Im Rahmen der Studienleistung analysieren Sie eine selbstgewählte Zeitreihe, siehe dazu auch das erste Aufgabenblatt im \href{https://moodle.uni-bielefeld.de/course/view.php?id=1035}{Lernraum der PÜ}. Dieses Dokument listet Fragen auf, die Sie mit Ihrer Analyse beantworten können. Die Fragen sind gruppiert nach der Vorlesung, in der Sie die entsprechenden Verfahren kennengelernt haben. Sie werden fortlaufend ergänzt.

\paragraph{Vorlesung 1 (12.04.2023)}
\begin{itemize}
    \item Wer hat Ihren Datensatz erhoben und was für ein Ziel soll wohl damit verfolgt werden?
    \item Wie wählen Sie die Argumente \texttt{start}, \texttt{end} und \texttt{frequency}, wenn Sie in \texttt{R} ein Zeitreihenobjekt mit der Funktion \texttt{ts()} erstellen? Wie können Sie Ihre Zeitreihe grafisch darstellen?
\end{itemize}