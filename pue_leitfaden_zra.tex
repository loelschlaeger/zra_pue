\begin{center}
    \Large\sffamily
    \textbf{Leitfaden zur Zeitreihenanalyse}
\end{center}

Im Rahmen der Studienleistung analysieren Sie eine selbstgewählte Zeitreihe, siehe dazu auch das erste Aufgabenblatt im \href{https://moodle.uni-bielefeld.de/course/view.php?id=1035}{Lernraum der PÜ}. Dieses Dokument listet Fragen auf, die Sie mit Ihrer Analyse beantworten können. Die Fragen sind gruppiert nach der Vorlesung, in der Sie die entsprechenden Verfahren kennengelernt haben. Sie werden fortlaufend ergänzt.

\paragraph{Vorlesung 1 (14.04.2023)}
\begin{itemize}
    \item Wer hat Ihren Datensatz erhoben und was für ein Ziel soll wohl damit verfolgt werden?
    \item Wie wählen Sie die Argumente \texttt{start}, \texttt{end} und \texttt{frequency}, wenn Sie in \texttt{R} ein Zeitreihenobjekt mit der Funktion \texttt{ts()} erstellen? Wie können Sie Ihre Zeitreihe grafisch darstellen?
\end{itemize}

\paragraph{Vorlesung 3 (28.04.2023)}
\begin{itemize}
    \item Aus welchen Komponenten besteht das klassische Komponentenmodell und welche davon finden Sie rein visuell in Ihrer Zeitreihe wieder?
    \item Wie können Sie den Trend einer Zeitreihe durch Regression modellieren? 
    \item Wie sieht Ihre Zeitreihe nach einer Trendbereinigung durch Differenzenbildung aus?
\end{itemize}

\paragraph{Vorlesung 4 (05.05.2023)}
\begin{itemize}
    \item Wie wählen Sie den Grad einer polynomialen Trendmodellierung mittels F-Test, Modellselektionskriterien, oder Kreuzvalidierung? Variiert die jeweilige Empfehlung?
    \item Gibt es Strukturbrüche in Ihrer Zeitreihe, die Sie entweder visuell erkennen oder durch Ereignisse bedingt sein können? Wie können Sie diese Strukturbrüche modellieren und auf Signifikanz testen?
\end{itemize}
