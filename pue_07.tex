\aufgabenblatt{7}

\aufgabe{1}{Modellselektion}

Sie finden im \href{https://moodle.uni-bielefeld.de/course/view.php?id=1035}{Lernraum der PÜ} die Datei \texttt{secret.csv}, die eine Zeitreihe enthält.

\begin{enumerate}

\item Bitte führen Sie eine Trend- und Saisonbereinigung der Zeitreihe durch.

\comment{
Visuell setzen wir die Frequenz auf 12 fest. Mit \texttt{decompose()} eliminieren wir am einfachsten Trend und Saison, siehe \texttt{R} Code.
}

\item Untersuchen Sie die verbleibende Komponente mithilfe von ACF und PACF. Können Sie sich auf dieser Basis für ein MA- oder AR-Modell entscheiden?

\comment{
Die ACF klingt geometrisch ab, die PACF ist nach Lag 3 nicht mehr signifikant von null verschieden. Beides deutet auf einen AR(3) Prozess hin, siehe \texttt{R} Code.
}

\item Passen Sie mit der \texttt{arima()} Funktion unterschiedliche MA($q$)- und AR($p$)-Modelle für $p,q \in \{1,2,3,4\}$ an und führen Sie eine Modellselektion mittels AIC durch. 

\emph{Hinweis:} Der Listenoutput von \texttt{arima()} hat einen Eintrag \texttt{aic} mit dem AIC-Wert.
\end{enumerate}

\comment{
Das AIC ist am kleinsten für das AR(4)-Modell, siehe \texttt{R} Code.
}

\aufgabe{2}{Signifikanztest für Autokorrelation}

Betrachten Sie erneut den Geburten Datensatz, den Sie bereits von Aufgabenblatt 4 kennen.

\begin{enumerate}

\item Schätzen Sie für den Zeitraum Januar 2000 bis einschließlich Februar 2020 ein gemeinsames Trend- und Saisonmodell.

\comment{
Im \texttt{R} Code wurde ein lineares Regressionsmodell mit Zeitindex im Grad 1, 2 und 3 und Monatsdummies geschätzt.
}
 
\item Schätzen Sie für die Modellresiduen ein adäquates zyklisches Modell.

\comment{
Mit Blick auf die ACF und PACF habe ich mich für ein AR(2)-Modell entschieden.
}

\item Führen Sie mit \texttt{Box.test()} einen Box-Pierce und einen Ljung-Box Test zum Signifikanzniveau 10\% durch, um zu überprüfen, ob nach der Zyklusmodellierung noch signifikante Autokorrelation bis zur 5. Ordnung vorhanden ist.

\comment{
Gemäß Box-Pierce nein, gemäß Ljung-Box ja, siehe \texttt{R} Code.
}

\end{enumerate}

\aufgabe{3}{Prognose}

\begin{enumerate}

\item Betrachten Sie erneut den Geburten Datensatz aus Aufgabe 2 bis einschließlich Februar 2020. Welche Vermutungen haben Sie über den weiteren Zeitreihenverlauf?

\comment{
Die Saison wird wohl fortgesetzt, der Trend nimmt wohl ab.
}

\item Berechnen Sie die Prognose der in Aufgabe 2b) bestimmten zyklischen Komponente für März 2020 bis einschließlich Februar 2023. 

\comment{
Siehe \texttt{R} Code.
}

\item Plotten Sie den gesamten Verlauf der Geburten Zeitreihe. Fügen Sie dann gefittete Werte und die Prognose aus dem Komponentenmodell ein.

\comment{
Siehe \texttt{R} Code.
}

\end{enumerate}