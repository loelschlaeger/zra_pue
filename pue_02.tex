\aufgabenblatt{2}

\aufgabe{1}{Kennzahlen}

\begin{enumerate}

\item Die Schlusskurse von Bitcoin und Ethereum aus $2022$ haben eine Kovarianz von $7718820$. Kann daraus ein starker, positiver Zusammenhang geschlossen werden?

\comment{
Eine positive Kovarianz deutet einen positiven Zusammenhang an. Da die Kovarianz aber nicht normiert ist, bleibt die Stärke des Zusammenhanges unbekannt.
}

\item Finden Sie ein Beispiel, dass im Allgemeinen $\Var (X + Y) \neq \Var (X) + \Var (Y)$ für zwei Zufallsvariablen $X$ und $Y$ gilt.

\comment{Sei $X = Y \sim N(0, 1)$. Dann ist $\Var (X) = \Var (Y) = 1$, aber $\Var (X+Y) = 4$.}

\item Finden Sie zwei Zufallsvariablen, die unkorreliert, aber nicht unabhängig sind.

\comment{
Wähle $X,Y$ unabhängig mit $\Pr(X=0)=\Pr(X=1)=0.5$, $\Pr(Y=-1)=\Pr(Y=1)=0.5$ und definiere $Z := XY$. Dann ist $\Cov (Z,X) = \E (Z(X-0.5)) = \E(X^2-0.5X) \E(Y)=0$ aber $\Pr(Z=1\mid X = 0) = 0 \neq 0.5 = \Pr(Z=1\mid X = 1)$.
}

\item Finden Sie zwei reelle Vektoren $x$ und $y$, jeweils der Länge $10$, sodass die empirische Korrelation $\widehat{\text{Cor}} (x,y)$ exakt $-1$ bzw. $0$ bzw. $+1$ beträgt.

\comment{
Für $x = 1,\dots, 10$ und $y = x$ ist $\widehat{\text{Cor}} (x,y) = +1$. Für $y = -x$ ist $\widehat{\text{Cor}} (x,y) = -1$. Für $y$ mit $y_i = (x - \bar{x})^2$, $i = 1,\dots,10$ ist $\widehat{\text{Cor}} (x,y) = 0$. 
}

\item Es seien $X,Y$ zwei unabhängige Zufallsvariablen, die mit gleicher Wahrscheinlichkeit die Werte in $\{1, 2, 3\}$ annehmen. Berechnen Sie $\E (1 + 4X + 2Y \mid X = 2)$.

\comment{
$\E (1 + 4X + 2Y \mid X = 2) = 9 + 2\E (Y) = 13$
}
    
\end{enumerate}

\aufgabe{2}{Schließende Statistik}

Ein Losverkäufer behauptet, dass mindestens $20\%$ seiner Lose Gewinne seien. Die Käufer aber vermuten, dass der Anteil geringer ist. Es werden $n = 100$ Lose überprüft. 

\begin{enumerate}

\item Führen Sie einen statistischen Test zum Signifikanzniveau $\alpha = 5\%$ zur Streitschlichtung durch, wobei Sie die Aussage des Losverkäufers als Nullhypothese wählen.

\comment{
Der Anteil an Gewinnlosen sei binomialverteilt zu $n = 100$ und $p = 20\%$. Die Wahrscheinlichkeit, höchstens $14$ Erfolge zu haben, beträgt somit $8\%$, und höchstens $13$ Erfolge zu haben, $4.7\%$. Zu $\alpha = 5\%$ geben wir also dem Losverkäufer recht, falls in der Stichprobe mindestens $14$ Gewinne sind, ansonsten den Käufern. Siehe \url{https://www.youtube.com/watch?v=MBf9Iin6bpg} für mehr Details.
}

\item Angenommen, der wahre Anteil an Gewinnlosen beträgt nur $10\%$. Berechnen Sie die Wahrscheinlichkeit, mit der Ihr Test dem Losverkäufer fälschlicherweise recht gibt.

\comment{Das ist die Wahrscheinlichkeit, unter der Binomialverteilung mit $n = 100$ und $p = 10\%$ mindestens $14$ Erfolge zu beobachten. Sie beträgt $12.4\%$.
}

\item Sie möchten die in b) berechnete Wahrscheinlichkeit auf unter $5\%$ reduzieren. Auf welche Werte müssen Sie dafür entweder $\alpha$ oder $n$ verändern?

\comment{
Entweder $\alpha$ auf mindestens $13\%$ oder $n$ auf mindestens $135$ erhöhen.
}

\end{enumerate}

\aufgabe{3}{Vorhersagen}

\begin{enumerate}

\item Überlegen Sie, wie präzise Folgendes vorhersagbar ist: der morgige Bitcoinpreis, die Lebensdauer einer Batterie, ein Münzwurf, die nächste Sonnenfinsternis, Ihr späteres Gehalt, das Wetter am nächsten Wochenende.

\comment{
Der morgige Bitcoinpreis ist nicht bis kaum vorhersagbar. Die Lebensdauer einer Batterie ist mit einer gewissen Unsicherheit vorhersagbar. Das Ergebnis eines Münzwurfs ist nicht vorhersagbar. Der Zeitpunkt der nächsten Sonnenfinsternis ist exakt vorhersagbar. Mein späteres Gehalt ist mit einer gewissen Unsicherheit vorhersagbar, ebenso das Wetter am nächsten Wochenende.
}

\item Beschreiben Sie drei Alltagssituationen, in denen Sie Vorhersagen treffen. Warum glauben Sie, dass diese Situation prognostizierbar sind? Welche Faktoren führen zu Prognosefehlern? Welchen Wert haben gute im Vergleich zu schlechten Vorhersagen für Sie?

\comment{
Zum Beispiel das benötigte Geld für den nächsten Supermarkteinkauf: meine Einkaufsliste hilft, spontane Einkäufe oder Preisänderungen führen zu Fehlern, eine gute Vorhersage hilft bei der Haushaltsplanung, eine Unterschätzung möglicherweise zu Engpässen am Monatsende.
}

\item Erklären Sie anhand eines Beispiels, dass falsche Vorhersagen unterschiedliche Konsequenzen nach sich ziehen können.

\comment{
Abflugzeit: zu früh eingeschätzt ist meistens kostengünstiger als zu spät. 
}

\item Welche Charakteristiken weisen Zeitreihen auf, die leichter beziehungsweise schwieriger vorhersagbar sind?

\comment{
Bei der Vorhersage hilft eine Struktur in der Zeitreihe, wie Trend und Saisonalität, oder ein starker Zusammenhang mit einer anderen Variable, die beobachtbar ist.
}

\end{enumerate}
